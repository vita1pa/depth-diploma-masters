\chapter{Introduction and Asymptotic Analysis for Tukey Depth in the Bivariate Exponential Distribution}

\section{Introduction}

In this chapter, we study the Tukey (half-space) depth for a bivariate distribution with independent exponential marginals. Let
\[
X \sim \operatorname{Exp}(\lambda_1) \quad\text{and}\quad Y \sim \operatorname{Exp}(\lambda_2),
\]
with joint density
\[
f_{X,Y}(x,y)=\lambda_1\lambda_2 \, e^{-\lambda_1 x-\lambda_2 y},\quad x,y \ge 0.
\]
The Tukey depth of a point is defined as the minimum probability mass of any closed half-plane that contains the point. In our approach, we consider a family of lines passing through the point $(a,b)$ and study the probability mass below the line. (Note that in general the depth is given by 
\[
D(a,b)=\min\{P,1-P\},
\]
but here we focus on the computation of one side's probability as a function of the line parameters.)

\section{Definition of the Probability Mass \( I(k) \)}

Consider a line through $(a,b)$ given by
\begin{equation}\label{eq:line}
  y - b = k \, (x - a),
\end{equation}
where \( k = \tan\theta \). In the first quadrant, the region below this line forms a triangle. The \(x\)-intercept is found by setting \( y = 0 \):
\[
0 - b = k\,(x-a) \quad \Longrightarrow \quad x = a - \frac{b}{k}.
\]
We define
\[
x_0 = a - \frac{b}{k}, \quad \text{(assumed positive)}.
\]
Then, for each \( x \in [0, x_0] \), the corresponding \( y \) runs from 0 up to
\[
y_{\max}(x) = k(x-a)+b.
\]
Thus, the probability mass below the line is given by
\begin{equation}\label{eq:Ik}
  I(k) = \int_{x=0}^{x_0} \int_{y=0}^{k(x-a)+b} \lambda_1 \lambda_2 \, e^{-\lambda_1 x-\lambda_2 y} \, dy\, dx.
\end{equation}
Evaluating the inner integral,
\[
\int_{0}^{k(x-a)+b} \lambda_2 e^{-\lambda_2 y} \, dy = 1 - e^{-\lambda_2\bigl[k(x-a)+b\bigr]},
\]
we obtain
\begin{equation}\label{eq:Ik2}
  I(k) = \int_{x=0}^{x_0} \lambda_1 e^{-\lambda_1 x} \Bigl[ 1 - e^{-\lambda_2\bigl(k(x-a)+b\bigr)} \Bigr] dx.
\end{equation}

\section{Differentiation via the Leibniz Rule}

Since the upper limit \(x_0 = a - \frac{b}{k}\) depends on \(k\), we differentiate \( I(k) \) using the Leibniz rule. In general, if
\[
I(k) = \int_{0}^{x_0(k)} f(x,k) \, dx,
\]
then
\[
\frac{dI}{dk} = \int_{0}^{x_0(k)} \frac{\partial f(x,k)}{\partial k}\, dx + f\bigl(x_0(k), k\bigr) \, \frac{dx_0}{dk}.
\]
In our case, 
\[
f(x,k) = \lambda_1 e^{-\lambda_1 x} \Bigl[ 1 - e^{-\lambda_2 \bigl(k(x-a)+b\bigr)} \Bigr].
\]
The \(k\)-dependence is entirely in the exponential term. Differentiating,
\[
\frac{\partial}{\partial k} \left[ 1 - e^{-\lambda_2 \bigl(k(x-a)+b\bigr)} \right] = \lambda_2 (x-a) \, e^{-\lambda_2\bigl(k(x-a)+b\bigr)}.
\]
Thus,
\[
\frac{\partial f(x,k)}{\partial k} = \lambda_1 e^{-\lambda_1 x} \, \lambda_2 (x-a) \, e^{-\lambda_2\bigl(k(x-a)+b\bigr)}.
\]
Also, differentiating \( x_0 = a - \frac{b}{k} \) with respect to \( k \) gives
\[
\frac{dx_0}{dk} = \frac{b}{k^2}.
\]
At \(x=x_0\),
\[
k(x_0-a)+b = k\Bigl(a-\frac{b}{k}-a\Bigr)+b = -b+b = 0,
\]
so that
\[
1 - e^{-\lambda_2\bigl(k(x_0-a)+b\bigr)} = 0.
\]
Hence, the boundary term vanishes and we obtain
\begin{equation}\label{eq:dIdk}
  \frac{dI}{dk} = \int_{x=0}^{x_0} \lambda_1 e^{-\lambda_1 x}\,\lambda_2 (x-a)\, e^{-\lambda_2\bigl(k(x-a)+b\bigr)}\, dx.
\end{equation}
Setting \(\frac{dI}{dk} = 0\) yields an implicit equation for the optimal \( k^* \).

\section{Asymptotic Analysis: Two Regimes}

We now discuss how the behavior of the function 
\[
g(x) = k(x-a)+b
\]
affects the optimal condition for \(k\) in two regimes.

\subsection{Small-Scale Regime}

In the regime where the values of \(g(x)\) remain small for all \(x \in [0,x_0]\), we can expand the exponential in a Taylor series:
\[
e^{-\lambda_2\,g(x)} \approx 1 - \lambda_2\,g(x).
\]
Then,
\[
1 - e^{-\lambda_2\,g(x)} \approx \lambda_2\,g(x) = \lambda_2 \bigl[k(x-a)+b\bigr].
\]
Substitute this into \eqref{eq:Ik2}:
\[
I(k) \approx \int_{0}^{x_0} \lambda_1 e^{-\lambda_1 x}\, \lambda_2 \bigl[k(x-a)+b\bigr] dx.
\]
Differentiating this approximate form with respect to \(k\) and setting the derivative equal to zero leads (after some algebra) to the condition
\[
k^* \approx \frac{\lambda_1}{\lambda_2}.
\]
Since \(k = \tan\theta\), this implies
\[
\tan\theta^* \approx \frac{\lambda_1}{\lambda_2}.
\]

\subsection{Large-Scale Regime}

When the values of \(g(x) = k(x-a)+b\) become large for most \(x\) in \([0,x_0]\) except near the upper limit \(x=x_0\) (where \(g(x_0)=0\)), the exponential \(e^{-\lambda_2\,g(x)}\) decays rapidly except in a narrow region near \(x_0\). To focus on this region, we define a new variable
\[
t = x_0 - x,
\]
so that when \(x=x_0\), \(t=0\) and when \(x=0\), \(t = x_0\). In terms of \(t\), we have
\[
x = x_0 - t.
\]
Then,
\[
g(x) = k\bigl((x_0-t)-a\bigr)+b = k(x_0-a)+b - k\,t.
\]
Since \(k(x_0-a)+b=0\), it follows that
\[
g(x) \approx -k\,t.
\]
(Here, the negative sign indicates that \(g(x)\) decreases linearly to 0 at \(t=0\); we are interested in the magnitude.) Hence, the exponential factor becomes
\[
e^{-\lambda_2\,g(x)} \approx e^{\lambda_2 k\,t}.
\]
Because \(e^{\lambda_2 k\,t}\) grows rapidly as \(t\) increases, and our original integrals in \eqref{eq:Ik2} involve \(1-e^{-\lambda_2\,g(x)}\), the dominant contribution arises from values of \(t\) very close to 0 (i.e. \(x\) near \(x_0\)), where the approximation \(g(x) \approx k\,t\) holds. Using Laplace's method, we approximate the integral by evaluating the slowly varying factor \( \lambda_1e^{-\lambda_1 x} \) at \(x=x_0\) and integrating the exponential term over \(t\):
\[
\int_{t=0}^{\epsilon} e^{\lambda_2 k\,t}\, dt \approx \frac{e^{\lambda_2 k\,\epsilon}-1}{\lambda_2 k},
\]
for a small \(\epsilon\). In the limit where the decay is very rapid (large \(\lambda_2 k\)), the effective support is very near \(t=0\). Differentiating the logarithm of the dominant contribution with respect to \(k\) yields a correction term of order \(1/k\). Expressing this condition in terms of \(\theta\) (since \(k=\tan\theta\)) leads to an optimality condition that includes a logarithmic correction. (A detailed derivation shows that the condition takes the form)
\[
\tan\theta^* \approx \frac{\lambda_1}{\lambda_2}\left(1+\Delta\right),
\]
with
\[
\Delta \sim \frac{1}{K}\ln\frac{\lambda_1}{\lambda_2},
\]
where \(K\) represents the characteristic scale of \(g(x)\) in the region near \(x_0\). In a fully detailed derivation, this correction is found to be of the form
\[
\tan\theta^* \approx \frac{\lambda_1}{\lambda_2}\left(1+\frac{1}{c}\ln\frac{\lambda_1}{\lambda_2}\right),
\]
if one were to reintroduce the geometric scale \(c\). However, since we are not introducing a separate projection variable, the key takeaway is that in the large-scale regime the optimal \(k^*\) deviates from \(\lambda_1/\lambda_2\) by a logarithmic correction that arises from the steep decay of the exponential term in a small neighborhood near \(x=x_0\).

\section{Summary}

To summarize, working solely in terms of \(k\), \(a\), and \(b\):
\begin{itemize}
  \item The integration region is given by \(x\in[0, x_0]\) with \(x_0 = a-\frac{b}{k}\) and, for each \(x\), \(y\in[0,\,k(x-a)+b]\).
  \item The probability mass below the line is 
  \[
  I(k)=\int_{x=0}^{x_0}\lambda_1e^{-\lambda_1x}\Bigl[1-e^{-\lambda_2\bigl(k(x-a)+b\bigr)}\Bigr]dx.
  \]
  \item Differentiating \(I(k)\) with respect to \(k\) using the Leibniz rule yields an implicit equation for the optimal \(k^*\).
  \item In the regime where \(k(x-a)+b\) remains small (small-scale regime), a Taylor expansion shows that \(k^*\approx\frac{\lambda_1}{\lambda_2}\).
  \item In the regime where \(k(x-a)+b\) is large except near \(x=x_0\) (large-scale regime), by substituting \(t=x_0-x\) and applying Laplace's method we find that the integral is dominated by a narrow region near \(t=0\). Differentiating the logarithm of this dominant contribution introduces a logarithmic correction in the optimal condition.
\end{itemize}

While the exact form of the logarithmic correction requires a full derivation, the key point is that in the large-scale regime the optimal slope \(k^*\) deviates from \(\lambda_1/\lambda_2\) by a term proportional to a logarithm of \(\lambda_1/\lambda_2\) (scaled by the characteristic size of the integration region near \(x_0\)). This completes our extended explanation entirely in terms of \(k\), \(a\), and \(b\).