\chapter{Projection of an Exponential Point and Hypoexponential Distribution}

In this chapter we consider the same joint density 
\[
f_{X_1,X_2}(x_1,x_2)=\lambda_1\lambda_2\, e^{-\lambda_1 x_1-\lambda_2 x_2},\quad x_1,x_2\ge0,
\]
and analyze the depth of a fixed point \((a,b)\) in \(\mathbb{R}^2\). Here, the random vector is denoted by 
\[
\mathbf{X}=(X_1,X_2),
\]
and we study the projection of \(\mathbf{X}\) onto an arbitrary direction.

\section{Projection onto a Given Direction and Half-Space Definition}

Let \(u=(\cos\theta,\sin\theta)\) be an arbitrary unit vector. The projection of the random vector \(\mathbf{X}=(X_1,X_2)\) onto \(u\) is given by
\[
Z = u^T \mathbf{X} = X_1\cos\theta + X_2\sin\theta.
\]
For a fixed point \((a,b)\), its projection is
\[
u^T(a,b)=a\cos\theta+b\sin\theta.
\]
Define the half-space associated with the direction \(u\) as
\[
H(u)=\{x\in\mathbb{R}^2 : \, u^T x \ge u^T(a,b)\}.
\]
Then the Tukey depth of \((a,b)\) is defined as
\[
D(a,b)=\inf_{u\in S^1}\Bigl\{P\bigl(x\in\mathbb{R}^2: \, u^T x \ge u^T(a,b)\bigr)\Bigr\} 
= \inf_{u\in S^1}\Bigl\{1-F\bigl(u^T(a,b)\bigr)\Bigr\},
\]
where \(F\) is the cumulative distribution function (CDF) of the projection \(Z\).

\section{Projection and the Hypoexponential Distribution}

Since \(X_1\) and \(X_2\) are independent exponential random variables, the projection
\[
Z = X_1\cos\theta + X_2\sin\theta
\]
can be viewed as a sum of two independent scaled exponentials. Define
\[
\mu_1 = \frac{\lambda_1}{\cos\theta} \quad \text{and} \quad \mu_2 = \frac{\lambda_2}{\sin\theta}.
\]
If \(\mu_1\neq \mu_2\), then \(Z\) follows a hypoexponential distribution with CDF
\[
F(z)= 1 - \frac{\mu_2}{\mu_2-\mu_1}\,e^{-\mu_1z} + \frac{\mu_1}{\mu_2-\mu_1}\,e^{-\mu_2z}, \quad z\ge0.
\]
In the special case when \(\mu_1=\mu_2=\mu\) (which occurs if \(\lambda_1\sin\theta=\lambda_2\cos\theta\)), the CDF becomes
\[
F(z)= 1 - (1+\mu z)e^{-\mu z}.
\]
Thus, the probability that \(\mathbf{X}\) falls in the half-space is
\[
P\{u^T \mathbf{X}\ge u^T(a,b)\} = 1 - F\bigl(u^T(a,b)\bigr),
\]
and the depth is
\[
D(a,b)= \inf_{u\in S^1}\Bigl\{ 1-F\bigl(a\cos\theta+b\sin\theta\bigr)\Bigr\}.
\]

\section{Summary and Optimization Approach}

The key idea is that the projection \(Z = X_1\cos\theta+X_2\sin\theta\) follows a hypoexponential distribution (with the special case \(\mu_1=\mu_2\) considered separately). Consequently, the probability mass in the half-space determined by \(u\) is 
\[
1-F\bigl(a\cos\theta+b\sin\theta\bigr),
\]
and the Tukey depth of \((a,b)\) is obtained by taking the infimum of this expression over all directions \(u\) (or equivalently, over \(\theta\)). In practice, one may evaluate
\[
1-F\bigl(a\cos\theta+b\sin\theta\bigr)
\]
explicitly using the hypoexponential CDF formulas provided above and then use numerical methods (such as grid search or optimization routines) to find
\[
D(a,b)= \inf_{\theta\in[0,2\pi)}\Bigl\{1-F\bigl(a\cos\theta+b\sin\theta\bigr)\Bigr\}.
\]

This formulation reduces the multidimensional depth problem to a one-dimensional optimization problem involving the hypoexponential CDF.

% End of chapter2.tex
